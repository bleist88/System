% v3 edited by GMW 2014.10.26 1730 EDT
% v5 edited by GMW 2014.10.27 2155 EDT
\documentclass[12pt]{article}
\usepackage{setspace}
\usepackage[pdftex]{graphicx}
\DeclareGraphicsExtensions{.pdf,.png}
\setlength{\parindent}{0em}
%\setlength{\voffset}{-2cm}
%\setlength{\hoffset}{-1cm}
\setlength{\headheight}{0.15in} % changed from 0.65 in, since lots of space is wasted at the tops of pages; GMW 2013.11.08 0413 EST
\setlength{\topskip}{0cm}
\setlength{\headsep}{0cm}
\setlength{\topmargin}{0cm}
\setlength{\oddsidemargin}{0in}
\setlength{\evensidemargin}{0.8in}
%\setlength{\textwidth}{17cm}
\setlength{\textheight}{9in}
\renewcommand{\rmdefault}{cmss}
\setlength{\textwidth}{6.5in}
\pagestyle{empty}
\begin{document}
\newcommand{\Msun}{{\rm M}$_\odot$}
\newcommand{\Lya}{{\rm Ly}$\sf \alpha$}
\newcommand{\mum}{$\sf \mu$m}



\def\spose#1{\hbox to 0pt{#1\hss}}
\def\simlt{\mathrel{\spose{\lower 3pt\hbox{$\mathchar"218$}}
   \raise 2.0pt\hbox{$\mathchar"13C$}}}
\def\simgt{\mathrel{\spose{\lower 3pt\hbox{$\mathchar"218$}}
   \raise 2.0pt\hbox{$\mathchar"13E$}}}
\def\lsim{\rlap{$<$}{\lower 1.0ex\hbox{$\sim$}}}
\def\gsim{\rlap{$>$}{\lower 1.0ex\hbox{$\sim$}}}
\def\etal{et al.}
\def\kms{km~s$^{-1}$}
\def\refs{\leftskip=.3truein\parindent=-.3truein}
\def\unrefs{\leftskip=0.0truein\parindent=20pt}

\vspace*{-1cm}

%\centerline
\begin{spacing}{1.5}
{\bf \Large The GALEUS Survey: Dust properties and morphologies}
{\bf \Large of UV-selected Galaxies at z~{\boldmath $\sim 2$}}
\end{spacing}
\vspace{0.5cm}

{\bf Abstract: } We propose an ultraviolet (UV) to far-infrared (FIR) 
study of a set of $\sf z\sim 2$ Lyman
Break Galaxies (LBGs) using wide-deep public data sets to: 
1) identify a sample of Lyman Break Galaxies (LBGs) and determine
distribution functions of their physical characteristics, and
2) place them in the context of current observations of LBGs at higher
redshift, in particular with our sample's contribution to the IR background.
%We propose to use a real Lyman break method to identify UV-   
%and IR-bright LBGs at $1.5<z<2.5$ in the Great Observatories Origin
%Survey South and North (GOODS-S and -N), COSMOS, and Extended Groth 
%Strip (EGS) for study and comparison with LBGs at other redshifts. 
Of special interest are the Herschel (100-500 $\mu$m) FIR data, since
they allow us for the first time to observe the dust peak emission
(rest frame 100-200~\mum ) at this epoch and derive extinction
properties for a large sample of LBGs. Together with morphological
information from high resolution HST observations, we will be
able to better constrain the sources making up the extragalactic IR
background. We will create a well-defined sample of
true LBGs at redshifts $\sf z\sim 2$  for future studies to allow for consistent
comparisons with higher $\sf z$ LBG samples, which will be useful to the
astronomical community.
\vspace{0.2cm}

%\section{Introduction}
\hspace{0.1in}
{\bf Introduction: } 
{\it Our goals are:\\                                              
1) to determine distribution functions for star-forming galaxy parameters
at this epoch, and \\                            
2) to place them in the context of current observations of star-forming
galaxies at higher redshift, in particular with our sample's
contribution to the IR background.}

\hspace{0.1in}
Determining the processes that drove the formation and evolution of 
the ensemble of galaxies is a cornerstone of NASA's SMD Origins
program. Consequently, theoretical progress came from large, detailed
galaxy simulations (e.g., the Millennium Simulation, Springel et
al. 2005). Observational advances came from wide-deep public
surveys (e.g., COSMOS, AEGIS, UKIDSS; Davis et al. 2007; Lawrence et
al. 2007) and new facilities (Gemini, VLT, ALMA).
One particular area which will give clues to the nature of galaxy
evolution involves the IR background, which reveals the signature of star
formation and dust.

\hspace{0.1in}
The energy spectrum of the optical-IR background looks like the spectral
energy distribution (SED) of our Galaxy or the Antennae, with
comparable terms for the optical and IR in $\sf \nu f_{\nu}$ (Devlin
et al. 2009). Either (a) the two regimes are \underline{distinct}, with low
obscuration galaxies producing optical flux and dusty (ultra)-luminous
IR (LIRG/ULIRG)
galaxies in the IR (e.g. Devlin et al. 2009), or (b) the optical and IR
contributions are \underline{related}, where late-type galaxies at all
redshifts make the optical+IR flux. This latter scenario is
supported by recent results from Wang et al. (2011), who
resolved submm detections in the GOODS-N public field into several sources.
To distinguish between these two models and determine the contribution
the IR background at moderate redshifts, we must first have a
well-constructed sample of UV-selected galaxies at $\sf z\sim 2$,
which contribute flux, but do not dominate the mass
(Haberzettl et al. 2012, Ly et al. 2011).

{\it We could then examine the UV luminosity per unit mass and how
  much extinction re-processes it into the IR, a key mechanism in
  producing the IR background.} 

\hspace{0.1in}
{\bf Why Lyman Break Galaxies:}
LBGs can be considered prototypes for strongly star-forming galaxies,
and are identified by the lack of flux at wavelengths $< 912$~\AA\ in broad-band
images.
At
$\sf z\gsim 3$ this break can be observed in ground-based observed data,
while at $\sf 1\lsim z \lsim 2.5$ \underline{space based UV
observations are needed}. The Lyman break provides an easy selection method
over most of the age of the
Universe (e.g. Bouwens et al. 2005 and references therein), which
makes LBGs so attractive for
comparative studies of galaxy properties over time.

\hspace{0.1in}
At $\sf z\sim 8-6$, LBGs are mainly faint and compact ($\sf \lsim
1$~kpc) (Bouwens et al. 2010b; Oesch et al. 2010).
At $\sf z\sim 3-4$, they tend to be more massive and contribute
more to the star formation rate (SFR, Shim et al. 2007).  By
$\sf z\sim 1$, they make up only $\sf \sim 11$\% of galaxies, and include both
red/dusty/massive and blue populations (Burgarella et al. 2006,
2007).

\hspace{0.1in}
{\it A key epoch at $\sf z=1-3$:}
At z~$\sim 1-3$, , LBG SFRs decrease and dust
contents increase (Basu-Zych et al. 2011). Their morphology changes
from mostly proto-ellipticals to mainly disk galaxies, indicating that
somehow, angular momentum ``came into its own''.
Intriguingly, the SFR is also near its maximum
(Coe et al. 2013) and the
last significant star formation occurs in the most massive galaxies
(Kodama et al. 2007). 

\hspace{0.1in}
Recent observations reveal different populations of
galaxies such as sub-mm galaxies (SMGs) at $\sf z>2$, LBGs at
$\sf z\gsim3$, and old (distant red galaxies) and dusty (BzKs)
galaxies at $\sf 2\lsim z\lsim 3$, with some suggested overlap between them.
These {\it overlap galaxies}\, tend to have comparable masses and
dust contents, and represent more dusty and intensively star-forming
LBGs. Herschel data for $\sf z\sim 3$ LBGs indicate an overlap
population of ULIRGS  (Magdis et al. 2010), and an under-prediction of
IR luminosity based on UV fluxes (Rigopoulou et al. 2010). 

\hspace{0.1in}
{\bf LBGs as Tracers of Star Formation and Galaxy Evolution:}
LBGs generally represent 
star-forming, relatively dust-free, intermediate-mass
galaxies, albeit with evidence
for a high mass tail.

\hspace{0.1in}
\underline{LBGs in the mid-IR:} Some LBGs are
mid-IR-luminous  ($3.6-24$~\mum), corresponding to rest-frame near-IR (NIR)
at z~$\sim 1.5-3$ and observable by Spitzer+IRAC. 
Rest-frame NIR flux reveals stellar masses
(Cole et al. 2001; Glazebrook et
al. 2004), and some massive LBGs are revealed by high IRAC fluxes.
The 24~\mum\ MIPS data probe Polycyclic Aromatic
Hydrocarbons (PAHs) at $\sf z\sim 2$, from two rest-frame emission
features at 6.2~\mum\ and 7.7~\mum. 
Bright LBGs (L$>$L$^*$) and cold SCUBA sources show PAH emission 
(Reddy et al. 2006), and are considered
comparable to local starburst galaxies
(Siana et al. 2008, 2009).

\hspace{0.1in}
\underline{z~$\sim 2-3$ Infrared-luminous (IL)LBGs in the far-IR -- recent studies}:
\underline{\bf (A)} Magdis et al. (2010) used Herschel/PACS
at 100,160~\mum\ for a stacked sample of Spitzer-detected
$\sf z\sim 3$ LBGs in GOODS-N, and found luminosities
$\sf \langle L_{IR}\rangle \sim 1.6\times 10^{12}M_\odot$, 
corresponding to ULIRGS.
The MIPS-selected sample shows warmer dust temperatures than sub-mm
selected galaxies, and could be missed in sub-mm surveys
biased toward colder systems.
\underline{\bf (B)} Rigopoulou et al. (2010) stacked Herschel/SPIRE images of
$\sf z\sim 3$ LBGs and $\sf z\sim 2$ BM/BX galaxies (a proxy to ground-select LBGs) 
in GOODS-N, and found fluxes 
$\sf \langle S_{250,LBG}\rangle =5.9\pm 1.4$~mJy 
with a fainter $\sf \langle S_{250,BM/BX}\rangle =2.7\pm 0.8$~mJy.  
The predicted $\sf L_{IR,BM/BX}$
is consistent with their UV data.  However,
the $\sf z\sim 3$ LBG UV fluxes
underpredict $\sf L_{IR,LBG}$, suggesting
that differences in physical properties 
(morphologies, dust content, star-forming regions)
between their LBGs and BM/BXs may be
significant.
\underline{\bf (C)} Ote et al. (2013) published a study of PACS detected $\sf z\sim 1$
LBGs. Their results indicate a trend between $\sf z\sim 3$ and $1$
from strong star forming LIRG/ULIRG type galaxies at $\sf z\sim 3$ to
normal star forming galaxies at $\sf z\sim 1$, which is supported
by our preliminary results, too (Fig.~1).  


\hspace{0.1in}
{\bf This Project in Brief:} We propose the 
GALEUS (GALaxy Evolution UV Survey) to use
public wide-deep surveys (Table~1) to study the physical
properties of UV-selected star-forming galaxies at $\sf z\sim 2$. We will
investigate the contribution of UV and IR luminous galaxies to the
population of LBGs, using UV to FIR data from a number of public wide-deep
fields (Table~1). 
{\it Detailed analysis of the galaxy SEDs and HST images will
let us meaningfully constrain the types of galaxies
contributing to the IR background, and construct a very useful true
LBG sample for future studies (e.g. by JWST).}

\hspace{0.1in}
\underline{\it The need for UV data}:  These identify 
true Lyman break galaxies at
$\sf z\sim 2$; see Haberzettl et al. (2012) for a first study for GOODS-S. 


\hspace{0.1in}
\underline{\it The need for mid/far-IR data}:  These 
will give us unique access to galaxy parameters like stellar
masses (IRAC 3.6-8.0 \mum) and dust properties (MIPS 24~\mum, Herschel
100 - 500~\mum) for an IR-bright LBG subsample. The Herschel data are
crucial, since they measure the dust emission peak
at rest-frame 100-200~\mum .
The dust properties and galaxy masses are
essential for understanding the evolution of star formation,
especially the reprocessing of UV-photons due to dust.

\hspace{0.1in}
\underline{\it The need for HST images}: We can then estimate morphologies of the sample galaxies. {\it If
  the background is mainly produced by late-type galaxies at all $\sf
  z$, with similar optical:IR contributions, we expect to identify a
  large fraction of spiral morphologies as well as spiral-spiral mergers.} 

\hspace{0.1in}
\underline{Additional data}: We will also use public 
broad-band optical/IR and spectroscopy data
(e.g. $\sim$20000 redshifts from DEEP2, Davis et al. 2007), 
to estimate accurate photo-$\sf z$'s.

\hspace{0.1in}
Our results will provide the base for future detailed
spectroscopic studies of subsamples of \underline{large numbers} of
\underline{bright} LBGs, using IFU and NIR spectroscopes. 
{\it It therefore complements deeper studies of much smaller fields
  with narrower NUV filters done by HST (e.g. Hathi et al. 2010),
  which can never cover the amount of area observed by GALEX.  It also
  builds on wide-field studies such as Haberzettl et al. (2012) and Ly
  et al. (2011).} 

\hspace{0.1in}
{\bf Preliminary Results:} A pilot study of a
subsample of IR-luminous LBGs (ILLBGs) in the GOODS-S shows the
feasibility of this project. We measured FIR (70--500~$\sf
\mu$m) fluxes for 5 ILLBGs using Herschel data from the surveys of Magnelli
et al. (2013) and Smith et al. (2012). The sample has
stellar masses of $\sf 3-20\times10^9$~M$_\odot$ and SFRs of
up to 200 M$_\odot$ yr$^{-1}$. Fitting modified Planck curves to the
Herschel FIR fluxes (Fig.~1), we estimated
dust temperatures of $\sim$45-60 K, dust masses between 
$\sf 1-300\times 10^7$~M$_\odot$, and total IR luminosities of
$\sf 2-30\times 10^{10}$L$_\odot$. We used HST images to
constrain radii from a few kpc for the compact objects to
$\sf \sim 14$~kpc for the extended disks. The Herschel-detected sample of 
ILLBGs exhibits properties comparable to normal star-forming 
rather than LIRG/ULIRG type galaxies, and also shows extended disk-like
morphologies (Fig.~1). These results support the
assumption that the IR background could be the result of late-type
galaxies at all redshifts. {\it However, the sample size is small.} 

\begin{table}[htbp]
\begin{tabular}{lll}
Filter Band&Field / References$^{5)}$&Field Size [deg$^2$]\\
\hline
GALEX FUV&GOODS-S, GOODS-N, COSMOS, EGS&1,1,2,1\\
GALEX NUV&GOODS-S, GOODS-N, COSMOS, EGS&1,1,2,1\\
UBVRIz&GOODS-S, GOODS-N&0.25,0.25\\
ugriz&COSMOS, EGS&1,1\\
HST ACS F606W/850LP&GOODS-S,GOODS-N,EGS$^{1)}$&0.25, 0.06,0.1\\
HST ACS F814W&COSMOS,EGS&1,0.17\\
NIR JHKs&GOODS-S,GOODS-N$^{2)}$COSMOS$^{3)}$,EGS$^{3)}$&0.06,0.1,2,0.7\\
Spitzer IRAC&GOODS-S, GOODS-N, COSMOS, EGS&0.4,0.4,1.7,0.8\\
Spitzer MIPS&GOODS-S, GOODS-N, COSMOS, EGS&0.4,0.7,1.7,0.8\\
Herschel PACS 70$^{4)}$,100,160\mum&GOODS-S,GOODS-N,COSMOS,EGS&0.4,0.04,2,0.2\\
Herschel SPIRE 250,350,500\mum&GOODS-S,GOODS-N,COSMOS,EGS&0.1,0.1,2,0.2\\
\hline
\end{tabular}
\caption{Overview of publicly available multi-frequency data for the
  proposed fields.
$^{1)}$ EGS only observed in HST ACS F606W;
$^{2)}$ combined H+K image;
$^{3)}$ only Ks data available;
$^{4)}$ only available for GOODS-S;
$^{5)}$ Scoville et al. 2007, Rix et al. 2004; Davis et
al. 2007; Koekemoer et al. 2007, Giavalisco et al. 2004;
Gawiser al. 2006; Ilbert et al. 2006; Capak et
al. 2004}
\end{table}



\begin{figure}[htbp]
\vspace*{-10mm}
\includegraphics[width=2.4in]{11800_1.png}
\hspace{0.1in} 
\includegraphics[width=1.8in]{sfr_compare_2.png}
\hspace{-0.1in} 
\includegraphics[width=1.8in]{mass_compare_v3.png}
%\end{figure}  
%\begin{figure}


%\hspace{0.1in}
%\includegraphics[width=2in, trim = 0in 0in 0in 1.7in, clip]{illbg08.png}
%\hspace{0.1in}
%\includegraphics[width=2in, trim = 0in 0in 0in 1.7in, clip]{illbg10.png}
\hspace{0.1in}
\includegraphics[width=3.0in, trim = 0in 0in 0in 1.7in, clip]{illbg13.png}
\caption{{\it Top Left}: Example for Herschel detected ILLBGs. The flux is
  fitted with modified Planck curves (red lines). The flux
  measurements included in the fit are indicated by black
  diamonds. Filter bands can be excluded from the fit (open diamonds)
  for two reasons a) target is not detected or b) flux of target is
  contaminated by neighboring objects. {\it Top middle}: Stellar
  mass vs. SFR -- The lines indicate the ``main sequence'' for normal star
  forming galaxies. {\it Top right}: Stellar Mass vs. dust mass --
  dust masses 
  of our ILLBGs are comparable to higher redshifted sub-mm
  galaxies. {\it Bottom row}: Results of a
  morphological analysis of HST-observed ILLBGs using GALFIT. {\it Left}:
  HST ACS F850LP (size $\sf 9^{\prime\prime}\times 9^{\prime\prime}$);
  {\it Middle}: best GALFIT model; {\it Right}: residual.}
\end{figure}

{\it Here we propose a more direct comparison with
  $\sf z\sim 2$ LBGs, the selection of which is not as strongly biased
  against IR-luminous galaxies as that for BM/BX galaxies (Haberzettl
  et al. 2012).}
Since LBGs represent the majority of the star formation rate at
$\sf z\simgt1$, they are our best tool available to address the following.

\hspace{0.1in}
{\bf (1)} We will constrain directly the types of
LBGs which contribute to the IR background. Are the contributions
in the optical from low obscuration galaxies, and in the IR from
LIRGS+ULIRGS?  Or do late-type galaxies produce the contribution
in both wavelength regimes (Wang et al. 2011)? This is
supported by our recent paper (Haberzettl et
al. 2012) and further preliminary results.

\hspace{0.1in}
{\bf (2)} Our LBG sample will lead to a number of follow-up studies, because
$\sf z\sim 2$ is a particularly interesting epoch
in terms of the star formation peak (Reddy et al. 2008, Coe et al. 2013), the
establishment of the morphology-density relation and
final star formation in massive galaxies  (e.g. Kodama et al. 2007).
We can address issues such as the escape fraction of Lyman continuum photons,
merger rate, distribution of metals and the detailed LBG mass
spectrum, among others.


\hspace{0.1in}
{\bf Specific goals:} This proposal is part of a larger on-going study
of LBGs at the crucial epoch of $\sf z \sim 2$.
%Here, we plan
%to study the general properties of LBGs, and specifically dust
%properties of ILLBGs in detail, using Herschel observations of the
%four survey fields.
As part of his Bachelor Thesis, B. Leist will address an important
technical problem in multi-wavelength studies: {\it ``How to efficiently
correlate object data from an inhomogeneous, multi-wavelength data
set?''}  
For the GALEUS project we have to combine information from up 20
observed filter-bands from the UV to mid-IR, with different field of
views, and different resolution for several hundred thousand objects.
The goal is to create a master object catalog with the highest
possible flexibility. To accomplish this goal Leist is currently
working on a python program (independent of licensed software packages like
IDL) for the multi-wavelength object correlation. The software is
designed to be easily adaptable for use in other projects with
different combinations of observed filter bands and will be provided
to the community with the first data release of GALEUS.  
Over the last few month he has already accomplished a promising preliminary
version and will finish a fully functioning version until his
graduation in May 2015. He is also expected to perform thorough tests
of the final version and help with the compilation of the final master
catalog and object selection under the lead of graduate student
Matthew Nichols (Graduate Student Fellowship proposal submitted separately). 

We expect the following specific goals by 05/15: (1) A final, fully
tested python program capable of compiling master catalogs from
inhomogeneous multi-wavelength data sets 3/15.
(2) The compilation of a master object catalog from catalogs created
via SExtractor searches (NUV to the 24~$\sf \mu$m) and selection of
LBG and ILLBG candidates following the methods described in Haberzettl
et al. (2012). 
Step (2) is in support for M. Nichols, who is applying for a separate
KSGC Graduate Fellowship.
A first publication of results from the GALEUS survey is expected in
the beginning of 2016. Further publications, especially on the
structural parameter and morphology of the sample galaxies are
expected to follow. 

\hspace{0.1in}
\underline{Anticipated results and importance::} We expect that Brian
Leist's python 
program will allow us to efficiently combine information from
multi-wavelength data sets. The tool will also add information about
the coverage of objects in the different filter-bands. For future
analysis it is important to know if objects are not covered by
observations in a filter-band, or if they are simply not detected. 
His results will be very valuable for the future work of our and other
groups. His support with compiling the master catalog and object
selection will help to speed up this first step of the project and
help M. Nichols' with his analysis.
A first publication on the overall properties of the
selected LBG and ILLBGs is expected for the beginning of 2016 and
Brian will be a co-author on this publication. We are also planning on
setting up a web-page for the GALEUS survey which will provide the
results, data releases (including catalogs and software tools) to the
public. The first data release is planed for December 2015. 

Even with the funding in period 2015, we expect that our
project will gain significantly. The results expected from the
proposed period in 2015 will help us when competing for other funding
resources to finish up additional steps of the project. For example we
will submit a HST archival proposal for the study of the morphologies
of star forming galaxies.
Additionally this study also has implication for future NASA missions
like JWST. Although the main focus of JWST is the high redshift
Universe, the IR properties of moderate redshifted star forming
galaxies are an important field of study and well characterized sample
of IR bright targets will benefit future JWST projects in this field. 


\hspace{0.1in}
\underline{Mentoring}: Haberzettl and Williger will advise Leist.
With his experience in the data analysis of multi-wavelength surveys
J. Colbert (IPAC) is the ideal NASA mentor, especially with Herschel
FIR data, and will have regular telecons with us.  Colbert 
is crucial for the successful implementation of the mid- to far-IR
data into the master catalogs. 



\hspace{0.1in}
\underline{Progress towards degree:} This proposal forms
the core of Leist's bachelor work. He will undergo advanced training
in multi-wavelength data mining. In the process, he will strengthen
his programming skills and help deriving object sample for further
analysis. Graduation is planned for May 2015.




\newpage
\noindent
{\bf References: }\\
Basu-Zych et al. 2011, ApJ 739, 98 \\                          %
Best 2004, MNRAS, 351,70\\                         %
Bouwens et al. 2004, ApJ, 616, L79\\               %
Bouwens et al. 2005, ApJ, 624, L5\\                %
Bouwens et al. 2010a, ApJ 709, 133\\               %
Bouwens et al. 2010b, ApJ 725, 1587\\              %
%Bundy et al. 2005, ApJ, 625, 621\\                 %
Burgarella et al. 2006, A\&A, 450, 69\\            %
Burgarella et al. 2007, MNRAS, 380, 986\\          %
Capak et al. 2004, AJ, 127, 180\\                  %
%Chapman et al.  2003, Nature, 422, 695\\
%Cimatti et al. 2002, A\&A, 2002, 381, 68\\
Coe et al. 2013, ApJ 762, 32\\                     %
Cole et al. 2001, MNRAS, 326, 255\\                %
Davis et al. 2007, ApJ, 660, L1\\                  %
Devlin et al. 2009, Nature, 458, 737\\             %
%Drory et al. 2004, ApJ, 608, 742\\                 %
Gawiser et al. 2006, ApJS, 162, 1\\                %
Giavalisco et al. 2004, ApJ, 600L, 93G\\           %
Glazebrook et al. 2004, Nature, 430, 181\\         %
%Guo \& White 2009, MNRAS, 396, 39\\
Haberzettl et al. 2012, ApJ\\                      %
Hathi et al. 2010, ApJ 720, 1708\\                 %
Ilbert et al. 2006, A\&A 457, 841\\                %
%Kere\v{s} et al. 2005, MNRAS, 363, 2\\
Kodama et al. 2007, MNRAS, 377, 1717\\             %
Koekemoer et al. 2007, ApJS, 172, 182\\            %
%Kriek et al.  2008, ApJ, 682, 896\\
Lawrence et al. 2007, MNRAS,  2007, 379, 1599\\    %
Lehnert \& Bremer 2003, ApJ, 593, 630\\            %
Lehnert et al. 2010, Nature, 467, 940\\            %
Ly et al. 2011, ApJ 735, 91L\\                     %
Magdis et al. 2010, ApJ, 720, L185\\               %
Magnelli et al. 2013, A\&A 553, 132\\              %
Oesch et al. 2010, ApJ 709, 21O\\                  %
%Ouchi et al. 2008, ApJS, 176, 301\\
Oteo et al. 2013, MNRAS,435,158\\
%Pei 1995, ApJ 438, 623\\                           %
Reddy et al. 2006, ApJ, 644, 792\\                 %
Reddy et al. 2008, ApJS, 175, 48\\                 %
Rigopoulou et al. 2010, MNRAS, 409, L7\\           %
Rix et al. 2004, ApJS, 152, 163\\                  %
Scoville et al. 2007, ApJS 172, 1\\                %
Shim et al. 2007, ApJ, 669, 749\\                  %
Siana et al. 2008, ApJ, 689, 59\\                  %
Siana et al. 2009, ApJ, 698, 1273\\                %
Smith et al. 2012, MNRAS 437, 703\\                %
Springel et al. 2005, Nature, 435, 629\\           %
%Steidel et al.  2003, ApJ, 592, 728\\              %
Wang et al. 2011, ApJ 726, 18\\                    %
%Wuyts et al.  2008, ApJ 682, 985\\    
%Yan et al. 2004, AJ 127, 1274

%\clearpage
%\setlength{\oddsidemargin}{-1in}
%\begin{figure}
%\vspace{-1in}
%\includegraphics[angle=-90]{10_28_13_Unofficial.pdf}
%\end{figure}

%\setlength{\oddsidemargin}{-1in}
%\begin{figure}
%\vspace{-1in}
%\includegraphics[page=1]{MTNResume.pdf}
%\end{figure}
%\setlength{\oddsidemargin}{-1in}
%\begin{figure}
%\vspace{-1in}
%\includegraphics[page=2]{MTNResume.pdf}
%\end{figure}
%\setlength{\oddsidemargin}{-1in}
%\begin{figure}
%\vspace{-1in}
%\includegraphics[page=1]{dbrown_support.pdf}
%\end{figure}
%\setlength{\oddsidemargin}{-1in}
%\begin{figure}
%\vspace{-1in}
%\includegraphics[page=2]{dbrown_support.pdf}
%\end{figure}
%\setlength{\oddsidemargin}{-1in}
%\begin{figure}
%\vspace{-1in}
%\includegraphics[page=1]{cv_gerry_ksgc201311.pdf}
%\end{figure}
%\setlength{\oddsidemargin}{-1in}
%\begin{figure}
%\vspace{-1in}
%\includegraphics[page=2]{cv_gerry_ksgc201311.pdf}
%\end{figure}


%\setlength{\oddsidemargin}{-1in}
%\begin{figure}
%\vspace{-1in}
%\includegraphics[page=2]{KYConsortium_20131107_1856_Nichols.pdf}
%\end{figure}

\end{document}
